 %!TEX TS-program = xelatex
%!TEX encoding = UTF-8 Unicode

%\def \papersize {a5paper}
\def \papersize {a4paper}
%\def \papersize {letterpaper}

%\documentclass[14pt,\papersize]{extarticle}
\documentclass[12pt,\papersize]{extarticle}
% extarticle is like article but can handle 8pt, 9pt, 10pt, 11pt, 12pt, 14pt, 17pt, and 20pt text

\def \ititle {Origins of Mind: Lecture Notes}
\def \isubtitle {Lecture 01}
%comment some of the following out depending on whether anonymous
\def \iauthor {Stephen A.\ Butterfill}
\def \iemail{s.butterfill@warwick.ac.uk% \& corrado.sinigaglia@unimi.it
}
%\def \iauthor {}
%\def \iemail{}
%\date{}

%\input{$HOME/Documents/submissions/preamble_steve_paper4}
\input{$HOME/Documents/submissions/preamble_steve_lecture_notes}

%no indent, space between paragraphs
\usepackage{parskip}

%comment these out if not anonymous:
%\author{}
%\date{}

%for e reader version: small margins
% (remove all for paper!)
%\geometry{headsep=2em} %keep running header away from text
%\geometry{footskip=1.5cm} %keep page numbers away from text
%\geometry{top=1cm} %increase to 3.5 if use header
%\geometry{bottom=2cm} %increase to 3.5 if use header
%\geometry{left=1cm} %increase to 3.5 if use header
%\geometry{right=1cm} %increase to 3.5 if use header

% disables chapter, section and subsection numbering
\setcounter{secnumdepth}{-1} 

%avoid overhang
\tolerance=5000

%\setromanfont[Mapping=tex-text]{Sabon LT Std} 


%for putting citations into main text (for reading):
% use bibentry command
% nb this doesn’t work with mynewapa style; use apalike for \bibliographystyle
% nb2: use \nobibliography to introduce the readings 
\usepackage{bibentry}

%screws up word count for some reason:
%\bibliographystyle{$HOME/Documents/submissions/mynewapa} 
\bibliographystyle{apalike} 


\begin{document}



\setlength\footnotesep{1em}






%--------------- 
%--- start paste
\title {Origins of Mind: Lecture Notes \\ Lecture 01}
 
 
 
\maketitle
 
 
 
\subsection{title-slide}
 
 
\section{The Question}
 
 
 
\subsection{slide-3}
This course is based on a simple question. The question is,
 
How do humans first come to know about---and to knowingly manipulate---objects, causes, words, numbers, colours, actions and minds?
 
We are going to approach this question by examining the evidence from developmental science, exploring how it bears on philosophical positions like nativism and empiricsm, and identifying philosophical problems created by the evidence.
 
 
 
\subsection{slide-4}
At the outset we know nothing, or not very much. (Like Lucas here.) Sometime later we do know some things. How does the transition occur?

 
 
 
\subsection{slide-5}
This is not a new question. There is a family of questions about the origins of mind that philosophers have been asking since Plato or before.
 
 
 
\subsection{slide-19}
Fodor mentions cognitive psychologists rather than philosophers. We will need to face up to the question of why philosophers are asking this question about the origins of knowledge, why is isn't just a scientific question. But that's something for later.
 
 
 
\subsection{unit\_021}
 
 
\section{From Myths to Mechanisms}
 
 
 
\subsection{myths\_to\_mechanisms}
How do humans come to know about---and to knowingly manipulate---objects, causes, words, numbers, colours, actions and minds?
 
In a beautiful myth, Plato suggests that the answer is recollection.
 
Before we are born, in another world, we become acquainted with the truth.
 
Then, in falling to earth, we forget everything.
 
But as we grow we are sometimes able to recall part of what we once knew.
 
So it is by recollection that humans come to know about objects, causes, numbers and everything else.
 
 
 
\subsection{slide-22}
Leibniz explicitly endorses a version of Plato's view.
 
The view is subtler than it seems: we'll return to the subtelties later.
 
 
 
\subsection{slide-23}
Locke, as you probably know, was an empiricist. Here's his manifesto.
 
(Here I'm contrasting Plato's and Leibnz' nativism with Locke's empiricism.)
 
 
 
\subsection{slide-24}
This claim about colour isn't relevant yet but we'll return to it later.
 
 
 
\subsection{slide-25}
Spelke is blunt.
 
Spelke doesn't have exactly Locke vs Leibniz in mind here, but it's close enough.
 
The quote continues ‘humans are endowed neither with a single, general-purpose learning system nor with myriad special-purpose systems and predispositions. Instead, we believe that humans are endowed with a small number of separable systems of core knowledge. New, flexible skills and belief systems build on these core foundations.’
 
 
 
\subsection{slide-29}
The claim that we should shift from thinking about myths to mechanisms raises two questions.
 
First, what are the mechanisms?
 
And, second, why suppose there is any role for philosophers rather than scientists?
 
 
 
\subsection{slide-26}
 
 
\section{Inbetween mindless behaviour and thought}
 
 
 
\subsection{slide-30}
Why suppose there is any role for philosophers rather than scientists? Part of the answer is provided by Donald Davidson.
 
The question is how humans come to know about objects, words, thoughts and other things.
 
In pursuing this question we have to consider minds where the knowledge is neither clearly present nor obviously absent.
 
This is challenging because both commonsense and theoretical tools for describing minds are generally designed for characterising fully developed adults.
 
 
 
\subsection{slide-39}
Hood and colleagues make a related point.
 
 
 
\subsection{slide-41}
This quote raises two issues
 
First, we should be cautious about the inference from separable systems to kinds of knowledge. (Think about modularity.)
 
 
 
\subsection{slide-43}
Second, we should be cautious here in talking about knowledge at all.
 
 
 
\subsection{slide-44}
To sum up so far, the question for this course is, How do humans come to know about---and to knowingly manipulate---objects, causes, words, numbers, colours, actions and minds?
 
I've been suggesting we can't answer it simply by appealing to nativism, empiricism or other grand myths.
 
Instead we need to focus on the particular mechanisms that are involved in different cases.
 
But then you might wonder, What philosophical questions arise here? Isn't this a narrowly pscyhological---and therefore scientific---issue?
 
The answer is no because thinking about how humans come to know things requires us to meet Davidson's challenge, to understand things that are neither mindless nor thought or knowledge but somewhere in between.
 
As Hood suggests in the quote I just showed you, this might involve rethinking what knowledge is.
 
--------
 
 
\subsection{slide-45}
We're going to try to understand how humans come to know about things by examining what developmental psychology tells us about the acquisition of knowledge.
 
This turns out to be a partly philosophical project because understanding the apparently conflicting evidence requires us to re-think notions like knowledge and representation.
 
In practice, this means looking carefully, and in detail, at the scientific evidence.
 
If you want to know how minds work, you have to start with the evidence.
 
 
 
\subsection{unit\_051}
 
 
\section{Two Breakthroughs}
 
 
 
\subsection{slide-47}
Recall that the question for this course is ...
 
Our focus is on two breakthrough sets of findings.
 
One concerns core knowledge, the other social interaction.
 
Take core knowledge first.
 
We will see that infants can tackle physics, number, agents and minds thanks to a set of innate or early-developing abilities, often labelled `core knowledge'.
 
 
 
\subsection{slide-48}
It is important that we don't (yet) know what core knowledge is; `core knowledge' is a term of art.
 
I haven't explained what it is.
 
Rather, our position is this.
 
Some scientists talk about core knowledge, and formulate hypotheses in terms of it.
 
Since these hypotheses are supported by evidence, we can reasonably suppose they are true.
 
So there are some things we suppose are truths about core knowledge.
 
For instance, infants' core knowledge enables them to represent unperceived objects.
 
So we know some truths about core knowledge, but we don't know what it is.
 
You'd probably prefer it if I could tell you what core knowledge is first.
 
But here we have to work backwards.
 
We have to gather truths about this unknown thing, core knowledge.
 
And then we have to ask, What could core knowledge be given that these things are true?
 
 
 
\subsection{slide-50}
Several features distinguish core knowledge from adult-like understanding: its content is unknowable by introspection and judgement-independent; it is specific to quite narrow categories of event and does not grow by means of generalization; it is best understood as a collection of rules rather than a coherent theory; and it has limited application being usually manifest in the control of attention (as measured by dishabituation, gaze, and looking times) and rarely or never manifest in purposive actions such as reaching.
 
 
 
\subsection{slide-53}
Now turn to social interaction.
 
Preverbal infants manfiest a surprising range of social abilities.
 
These include imitation, which can occur just days and even minutes after birth (Meltzoff \& Moore 1977; Field et al. 1982; Meltzoff \& Moore 1983), imitative learning (Carpenter et al. 1998), gaze following (Csibra \& Volein 2008), goal ascription (Gergely et al. 1995; Woodward \& Sommerville 2000), social referencing (Baldwin 2000) and pointing (Liszkowski et al. 2006).
 
Taken together, the evidence reveals that preverbal infants have surprisingly rich social abilities.
 
One problem for us is that these two sets of findings are typically considered in isolation, although I think there are strong reasons to suppose that understanding the origins of knowledge will require thinking about both core knowledge and social interaction.
 
My working hypothesis is that we can't understand early forms of social interaction without core knowledge; and that we can't understand how core knowledge leads to knowledge proper without social interaction.
 
The challenge is to understand how core knowledge and social interaction conspire in the emergence of knowledge.
 
 
 
\subsection{slide-54}
So the big challenge is to understand how core knowledge and social interaction combine to explain how humans first come to know things.
 
If you've seen the outline of lectures, you'll know that my approach is to work through what we know about different domains of knowledge.
 
 
 
\subsection{slide-55}
The domains are these.
 
 
 
\subsection{slide-59}
Today's topic is objects.
 
I've chosen to do this first because it's one of the best understood.
 
%--- end paste
%--------------- 
 





\bibliography{$HOME/endnote/phd_biblio}



\end{document}