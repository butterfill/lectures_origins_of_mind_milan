%!TEX TS-program = xelatex
%!TEX encoding = UTF-8 Unicode

\documentclass[12pt]{extarticle}
% extarticle is like article but can handle 8pt, 9pt, 10pt, 11pt, 12pt, 14pt, 17pt, and 20pt text

\def \ititle {Origins of Mind}
 
\def \isubtitle {Lecture 01}
 
\def \iauthor {Stephen A. Butterfill}
\def \iemail{s.butterfill@warwick.ac.uk}
\date{}

%for strikethrough
\usepackage[normalem]{ulem}

\input{$HOME/Documents/submissions/preamble_steve_handout}

%\bibpunct{}{}{,}{s}{}{,}  %use superscript TICS style bib
%remove hanging indent for TICS style bib
%TODO doesnt work
\setlength{\bibhang}{0em}
%\setlength{\bibsep}{0.5em}


%itemize bullet should be dash
\renewcommand{\labelitemi}{$-$}

\begin{document}

\begin{multicols}{3}

\setlength\footnotesep{1em}


\bibliographystyle{newapa} %apalike

%\maketitle
%\tableofcontents




%--------------- 
%--- start paste

\def \ititle {Origins of Mind}
 
\def \isubtitle {Lecture 01}
 
 
 
\
 
 
 
\begin{center}
 
{\Large
 
\textbf{\ititle}: \isubtitle
 
}
 
 
 
\iemail %
 
\end{center}
 
 
 
\section{The Question}
 
\textbf{Question}
How do humans first come to know about---and to knowingly manipulate---objects, causes, words, numbers, colours, actions and minds?
 
‘... ’tis past doubt, that Men have in their Minds several Ideas, such as are those expressed by the words, Whiteness, Hardness, ... and others: It is in the first place to be enquired, How he comes by them?’
\citep[p.\ 104]{Locke:1975qo}
 
‘How does it come about that the development of organic behavior into controlled inquiry brings about the differentiation and cooperation of observational and conceptual operations?’
\citep[p.\ 12]{Dewey:1938yp}
 
‘the fundamental explicandum, is the organism and its propositional attitudes ... Cognitive psychologists accept ... the ... necessity of explaining how organisms come to have the attitudes to propositions that they do.’
\citep[p.\ 198]{Fodor:1975pb}
 
 
 
\section{From Myths to Mechanisms}
 
‘the soul inherently contains the sources of various notions and doctrines which external objects merely rouse up on suitable occasions’
\citep[p.\ 48]{Leibniz:1996bl}
 
‘Men, barely by the Use of their natural Faculties, may attain to all the Knowledge they have, without the help of any innate Impressions; [...]
‘it would be impertinent to suppose, the Ideas of Colours innate in a Creature, to whom God hath given Sight, and a Power to receive them by the Eyes from external Objects’
\citep[p.\ 48]{Locke:1975qo}
 
‘Developmental science [...] has shown that both these views are false’
\citep[p.\ 89]{Spelke:2007hb}.
 
 
 
\section{Inbetween mindless behaviour and thought}
 
‘We have many vocabularies for describing nature when we regard it as mindless, and we have a mentalistic vocabulary for describing thought and intentional action; what we lack is a way of describing what is in between’ \citep[p.\ 11]{Davidson:1999ju}
 
‘there are many separable systems of mental representations ... and thus many different kinds of knowledge. ... the task ... is to contribute to the enterprise of finding the distinct systems of mental representation and to understand their development and integration’
\citep[p.\ 1522]{Hood:2000bf}.
  

%--- end paste
%--------------- 
 
\footnotesize 
\bibliography{$HOME/endnote/phd_biblio}

\end{multicols}

\end{document}