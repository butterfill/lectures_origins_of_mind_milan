%!TEX TS-program = xelatex
%!TEX encoding = UTF-8 Unicode

\documentclass[12pt]{extarticle}
% extarticle is like article but can handle 8pt, 9pt, 10pt, 11pt, 12pt, 14pt, 17pt, and 20pt text

\def \ititle {Origins of Mind}
 
\def \isubtitle {Lecture 08}
 
\def \iauthor {Stephen A. Butterfill}
\def \iemail{s.butterfill@warwick.ac.uk}
\date{}

%for strikethrough
\usepackage[normalem]{ulem}

\input{$HOME/Documents/submissions/preamble_steve_handout}

%\bibpunct{}{}{,}{s}{}{,}  %use superscript TICS style bib
%remove hanging indent for TICS style bib
%TODO doesnt work
\setlength{\bibhang}{0em}
%\setlength{\bibsep}{0.5em}


%itemize bullet should be dash
\renewcommand{\labelitemi}{$-$}

\begin{document}

\begin{multicols}{3}

\setlength\footnotesep{1em}


\bibliographystyle{newapa} %apalike

%\maketitle
%\tableofcontents




%--------------- 
%--- start paste


\def \ititle {Origins of Mind}
 
\def \isubtitle {Lecture 09}
 
 
 
\
 
 
 
\begin{center}
 
{\Large
 
\textbf{\ititle}: \isubtitle
 
}
 
 
 
\iemail %
 
\end{center}
 
 
 
\section{Action}
 
‘by the end of the first year infants are indeed capable of taking the intentional stance (Dennett, 1987) in interpreting the goal- directed behavior of rational agents.’
\citep[p.\ 184]{Gergely:1995sq}
 
‘12-month-old babies could identify the agent’s goal and analyze its actions causally in relation to it’
\citep[p.\ 190]{Gergely:1995sq}
 
'Six-month-olds and 9-month-olds showed a stronger novelty response (i.e., looked longer) on new-goal trials than on new-path trials (Woodward 1998). That is, like toddlers, young infants selectively attended to and remembered the features of the event that were relevant to the actor’s goal.'
\citep[p.\ 153]{woodward:2001_making}
 
‘just as the visual system works to recover the physical structure of the world by inferring properties such as 3-D shape, so too does it work to recover the causal and social structure of the world by inferring properties such as causality’
\citep[p.\ 299]{Scholl:2000eq}
 
‘in perceiving one object as having the intention of affecting another, the infant attributes to the object [...] intentions’
\citep[p.\ 14]{Premack:1990jl}
 
‘by taking the intentional stance the infant can come to represent the agent’s action as intentional without actually attributing a mental representation of the future goal state’
\citep[p.\ 188]{Gergely:1995sq}
 
‘to the extent that young infants are limited in this way, their understanding of intentions would be quite different from the mature concept of intentions’
\citep[p.\ 168]{woodward:2001_making}
 
 
 
\section{Intentions and Goals}
 
`The expression `the intention with which James went to church' has the outward form of a description, but in fact it
 ... cannot be taken to refer to an entity, state, disposition, or event. Its function in context is to generate new descriptions of actions in terms of their reasons; thus `James went to church with the intention of pleasing his mother' yields a new, and fuller, description of the action described in `James went to church'.'
\citep[p.\ 690]{davidson:1963_orig}
 
 
 
\section{Pure Goal Ascription: the Teleological Stance}
 
Csibra \& Gergely's principle of rational action: `an action can be explained by a goal state if, and only if, it is seen as the most justifiable action towards that goal state that is available within the constraints of reality.'\citep{Csibra:1998cx,Csibra:2003jv}
 
(Contrast a principle of efficiency:
`goal attribution requires that agents expend the least possible amount of energy within their motor constraints to achieve a certain end.' \citep%[p.\ 1061]
{Southgate:2008el})
 
These facts:
 
\begin{enumerate}
 
\item action $a$ is directed to some goal;
 
\item actions of $a$'s type are normally capable of being means of realising outcomes of $G$'s type in situations with the salient (to any concerned) features of this situation;
 
\item no alternative type of action is both typically available to agents of this type and also such that actions of this type would be normally be significantly better* means of realising outcome $G$ in situations with the salient features of this situation;
 
\item the occurrence of outcome $G$ is typically desirable for agents of this type;
 
\end{enumerate}
 
\begin{enumerate}[resume]
 
\item there is no other outcome, $G'$, the occurrence of which would be at least comparably desirable for agents of this type and where (2) and (3) both hold of $G'$ and $a$
 
\end{enumerate}
 
may jointly constitute defeasible evidence for the conclusion that:
 
\begin{enumerate}[resume]
 
\item $G$ is a goal to which action $a$ is directed.
 
\end{enumerate}
 
{
 
\footnotesize
 
*An action of type $a'$ is a better means of realising outcome $G$ in a given situation than an action of type $a$ if, for instance, actions of type $a'$ normally involve less effort than actions of type $a$
 
in situations with the salient features of this situation
 
and everything else is equal;
 
or if, for example, actions of type $a'$ are normally more likely to realise outcome $G$ than actions of type $a$ in situations with the salient features of this situation
 
and everything else is equal.
 
}
 
‘What it is to be a true believer is to be … a system whose behavior is reliably and voluminously predictable via the intentional strategy.’
\citep[p.\ 15]{Dennett:1987sf}
 
 

%--- end paste
%--------------- 
 
\footnotesize 
\bibliography{$HOME/endnote/phd_biblio}

\end{multicols}

\end{document}